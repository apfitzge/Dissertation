\chapter{Introduction}{\label{ch:Introduction}
    \section{Motivation}{\label{sec:Introduction:Motivation}
        Computer simulations have played an important role in the design and analysis of nuclear reactor systems over the past 60 years \cite{FewGroupDiffusion}.
        The methods used by these simulations have always been limited by the available computational resources; as such, in the 1950's two-group diffusion theory was used as a basis for simulation tools \cite{FewGroupDiffusion}.
        As computers became more powerful, multi-group diffusion calculations became the method of choice for \ac{LWR} design calculations.

        More accurate and detailed simulation tools allow for designs to have higher power density, and thus be more profitable, without compromising safety.
        However, computational resources have always limited the level of detail of simulation tools.
        Exponential increases in computing power, and high-performance computing clusters have made whole-core transport calculations possible \cite{CASMO-4,Apollo2-2010,DeCART,Denovo,Yang2010,Boyd2014,Collins2016,Gunow2018}.
        Programs such as \ac{CASL} and \ac{NEAMS} have focused on development of modern advanced simulation tools to address certain challenge problems.
        Large computing clusters are generally unavailable to reactor analysts in industry, and so using direct whole-core 3-D transport methods is not common outside academia or national laboratories.

        The ``gold standard'' of deterministic methods has been the 3-D \ac{MOC} \cite{Askew1972} due to its' ability to exactly model complicated geometries.
        At the time of writing, whole-core 3-D \ac{MOC} calculations are generally not possible without use of large computing clusters.
        This is due to the large discretizations that are necessary for the neutron transport equation, which has a 6-dimensional phase space for steady-state eigenvalue problems.
        In the past decade, there has been renewed interest in making 3-D \ac{MOC} more efficient and performant by using parallelism \cite{Kochunas2013}, modern \ac{GPU} architectures \cite{Boyd2014}, and ray-tracing storage techniques \cite{Sciannandrone2016, Gunow2016}.
        There has also been work done to make \ac{MOC} faster by improving the efficiency of the calculations by using higher-order approximations \cite{Ferrer2016,Gunow2018}.

        The bulk of this thesis work is comprised of three distinct, yet connected, topics, all with a focus on improving the feasibility of 3-D \ac{MOC} calculations.
        It is the author's opinion, that improving efficiency of 3-D \ac{MOC} calculations should be the primary focus of current research, as it is not feasible for industry to use thousands of processors.
        Thus, two techniques are utilized as part of this thesis work: the \acf{LSA}, and the macroray.

        The \ac{LSA} has been studied by other research groups \cite{Ferrer2016,Ferrer2018,Gunow2018}, and has been worked on as part of this thesis project; specifically, this work has led to improvements of the method for stability in near-void regions \cite{Fitzgerald2018}, and efficiency in multi-physics simulations \cite{Fitzgerald2019}.
        The \ac{LSA} is an approximation that is used to improve \ac{MOC} efficiency by reducing the number of computational cells required for accurate results.

        The macroray is a new ray-tracing technique under development as part of this thesis work; this technique is an extension of the two-dimensional macroband \cite{Villarino1992} ray-tracing technique.
        This technique has been shown to reduce the number of characteristic rays required for accurate results in two-dimensional flat-source calculations \cite{Yamamoto2005,Fevotte2007}.
        To the best of the author's knowledge, there have been no studies of this ray-tracing technique in three-dimensional ray-tracing calculations.
        Fewer characteristic rays results in more efficient calculations; the improvement in efficiency is expected to be more significant in 3-D calculations due to the square scaling of tracks with ray-spacing, rather than linear scaling in 2-D.
        Additionally, efficiency should be improved further by the \ac{LS} which allows for coarser cells and fewer track-segments.

        The third contribution of this thesis is work in improving parallel efficiency.
        While large scale parallelism on thousands of processors may not be feasible for industry, some degree of parallelism is necessary for whole-core calculations due to memory constraints.
        An automated spatial decomposition scheme based on graph theory, is developed leading to significantly improved parallel efficiency \cite{Fitzgerald2017,Fitzgerald2019a}.
    }
    \section{Outline}{\label{sec:Introduction:Outline}
        The remainder of this document is structured as follows.
        \Cref{ch:Neutron Transport Theory} gives an overview of neutron transport theory, with a focus on what is relevant to this work.
        The derivation and details on the \acf{MOC} are provided in \cref{ch:The Method of Characteristics}, with a focus on the contributions made in this work.
        Ray-tracing is an important aspect of \acf{MOC} calculations, and details about ray-tracing techniques are provided in \cref{ch:Ray-Tracing}.
    }
}