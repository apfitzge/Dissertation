\chapter{Summary, Conclusions, and Future Work}{
  \label{ch:Conclusions}
  This chapter provides a brief summary of the motivation for this work, and what was accomplished.
  The key findings for each study are listed.
  Also, possible paths for future research in these topics are given.

  \section{Summary of Work}{\label{sec:Summary of Work}
    % 1. Spatial Decomposition
    %   - Motivations:
    %     * Load imbalance -> inefficient parallel calculations
    %     * Load imbalance particularly bad if LS and large reflector region!
    %     * Unable to use arbitrary number of processors...
    %   - What was done:
    %     * Graph partitioning methods implemented for spatial decomp in MPACT
    %       (Why not METIS?)
    %     * 2-D methods studied
    %     * Different 3-D schemes studied
    % 2. Linear Source
    %   - Motivations:
    %     * Reduce transport cost (run-time, memory)
    %     * Reduced mesh -> other things (physics) are also cheaper
    %     * Unstable if exponential table is not large (F2) in problems with near-void regions
    %     * Inefficient in multiphysics (damn C coefficients)
    %   - What was done:
    %     * Use a modified exponential form: eliminates source of instability
    %     * Re-formulate LSA implicitly to eliminate C coefficients
    %     * Coarse mesh parameters were found for
    %       - Shown to be sufficient for large 3-D quarter core cycle depletion
    % 3. MacroRay
    %   - Motivations:
    %     * Primary driver of transport run-time is number of segments
    %     * Unable to use coarser rays if there are small strong absorber regions ANYWHERE in your problem
    %     * Macroband showed some promising results in 2-D (though not studied a lot)
    %     * No studies on advanced transverse integration in 3-D
    %   - What was done
    %     * A 3-D extension of the macroband was implemented (2-D/2-D ray-tracing approach)
    %     * Using sub-boundary averaging for treatment of interface and boundary flux
    %     * Tested on several 3-D transport benchmarks
    \subsection{Summary of Spatial Decomposition}{\label{ssec:Summary:Spatial Decomposition}
      Prior to this work, MPACT \cite{MPACT2016} used an automated spatial decomposition method based on the fuel assemblies.
      This method was limited, and only allowed for specific numbers of spatial domains to be used.
      Additionally, such methods have been shown, in other codes, to result in poor load balance, particularly when a large coarsely-meshed radial reflector is present \cite{Gunow2018}.
      Stimpson \cite{StimpsonPartitioning2017} improved MPACT's parallel capabilities, allowing for a problem to be decomposed into groups of ray-tracing modules.
      However, that work did not introduce a method for generating better spatial decompositions, and relied on the user to enter a decomposition map.
      This was tedious for the user, and ofter resulted in non-optimal decompositions.

      This dissertation's work on spatial decomposition sought to develop decomposition methods to
      \begin{enumerate}
        \item{improve load-balance,}
        \item{improve user experience (no additional work),}
        \item{and, allow for less restrictions on the number of domains.}
      \end{enumerate}
      These improvements were made by utilizing graph partitioning methods, which have been highly studied in computer science \cite{Elsner1997}.
      Software libraries exist for performing graph partitioning (METIS \cite{METIS}, Zoltan \cite{Boman2012}, etc.).
      However, due to MPACT's strict requirements on the decomposition (see \cref{sec:Spatial Decomposition:Spatial Decomposition in MPACT}), these libraries could not easily be used in MPACT.
      Several graph partitioning methods were implemented in MPACT.
      For unweighted graphs, and not conforming to MPACT's restrictions, these method were shown to be comparable to METIS \cite{Fitzgerald2017}, which is widely considered to give high quality partitions.

      The methods were then applied to MPACT and made to conform with the restrictions MPACT places on its spatial decomposition \cite{Fitzgerald2019a}.
      This work introduced a new graph partitioning algorithm based on recursive expansion, and an improvement to the Kernighan-Lin \cite{Kernighan1970} refinement for the application to the \ac{MOC}.
      These new methods, as well as the \acf{RSB} and \acf{RIB}, were used to decompose a 2-D transport problem, and resource comparisons of the simulations were presented.
      In 3-D, several different decomposition approaches were investigated, and a 3-D core was decomposed.
      For the 3-D calculations, only metrics of the decomposition were compared, rather than actual simulation results.
    }

    \subsection{Summary of Linear Source}{\label{ssec:Summary:Linear Source}
      Traditionally, \ac{MOC} codes have used the \acf{FSA}, however, more recently a few codes transitioned to using a \acf{LSA} \cite{Halsall1993,Petkov1999,Santandrea2002,Tang2009,Rabiti2009,Hebert2016,Ferrer2016,Gunow2018}.
      The general goal of these \acp{LSA} is to coarsen the computational mesh, leading to overall improved run-times.
      However, many of these previous \acp{LSA} have been \emph{ad-hoc}.
      Several of these methods \cite{Santandrea2002,Hebert2016,Ferrer2016} have been based on particle conservation.
      The \ac{LSA} developed by \citet{Ferrer2016} was implemented into the MPACT code \cite{Fitzgerald2018,Fitzgerald2019}.

      Once implemented in MPACT, the \ac{LSA} was tested on several problems, and two flaws were observed.
      The first was an implementation detail; typically, exponential functions in the \ac{MOC} are approximated to reduce run-time \cite{Yamamoto2004}.
      However, when the exponential functions in the \ac{LSA} were approximated in problems with near-void regions, such as the fuel-clad gap, iterative instability was observed.
      Different methods for addressing this instability were studied in this work \cite{Fitzgerald2018}.

      The second flaw was that for problems with multiple physics (\acf{TH} feedback, or isotropic depletion) or in the 2D/1D approximation \cite{Collins2016}, the method was far less efficient.
      This was due to the changing macroscopic cross sections, which led to coefficients needing to be recalculated at every state-point, or even every iteration.
      This work sought to eliminate the need to recompute these coefficients by eliminating their direct dependence on the cross section \cite{Fitzgerald2019}.
      The elimination of the group-dependent coefficients in the \ac{LSA} was achieved by re-formulating the method, as shown in\cref{ch:Improved Linear Source Formulation for Multi-physics and 2D/1D Applications}.
    }

    \subsection{Summary of MacroRay}{\label{ssec:Summary:MacroRay}
      Three-dimensional \ac{MOC} calculations are generally considered to be too expensive for use in routine design calculations or even reference solutions; the methods rarely see use outside academia or national laboratories.
      The primary factor in determining \ac{MOC} run-time is the number of track segments.
      By using a coarser ray-spacing, far fewer track-segments are generated; this can significantly improve run-times (\cref{sec:Improved Linear Source Formulation for Multi-physics and 2D/1D Applications}).
      However, the presence of small strong absorber regions anywhere in a problem can prevent the use of coarser rays.
      This motivates the use of a ray-tracing method that is able to use different effective ray-spacings in different regions with different requirements.

      The macroband \cite{Villarino1992,Petkov1999,Yamamoto2005,Fevotte2007} is a ray-tracing method that bases the placement of rays on the computational mesh of the problem.
      This method forces the use of finer ray-spacing where small regions are present, and allows for coarser ray-spacing where there are not small regions.
      With the use of non-uniform ray-spacing \cite{Yamamoto2005}, these methods can reduce the number of rays necessary for accurate integration.
      However, the method is not compatible with \acf{DNPL}, meaning that an approximation of the angular flux at interfaces is necessary, unlike the traditional \acf{MRT} ray-tracing methods.

      While the macroband has been previously studied for 2-D \ac{MOC}, there has been no extension or investigation of the method for 3-D \ac{MOC} calculations.
      This dissertation seeks to fill that gap, and provides one possible extension of the macroband to 3-D problems.
      This extension, the macroray ray-tracing method, was implemented in a library dependent on MPACT, and tested on several 3-D transport problems.
      The method was compared against the traditional \ac{MRT} ray-tracing method, with a particular focus on accuracy versus the number of track-segments.
    }
  }

  \section{Conclusions}{\label{sec:Conclusions}
    % 1. Spatial Decomposition
    %   - Greatly improved user experience:
    %     * Able to use any number of radial domains (must be same for each level in 2D/1D)
    %     * Automatic
    %     * Improves load-balance compared with manual decompositions
    %   - Load-balance greatly improved over current methods
    %   - Iterative efficiency decreased due to ``jagged'' domains (re-entrant rays)
    %   - CMFD solvers need to be improved...for large cases this becomes significantly more expensive than transport
    %     * Other groups have not seen this
    % 2. Linear Source
    %   - Exponentials:
    %     * Instability no longer observed for problems with near-void regions
    %     * Exponential and MOC run-times improved
    %   - Improved Formulation:
    %     * IMO: A more elegant form
    %     * Same number of operations (assuming FMA instructions are available)
    %     * Run-times improved by 10-15\% in multiphysics cases (T/H feedback)
    % 3. MacroRay
    %   - Uniform convergence behavior is nice
    %   - Much more of a ``black-box'' solver (I don't really need to know what my problem looks like)
    %   - Implementation is (unfortunately) inefficient:
    %     * However, this method has NOT had years of optimization like the MRT
    %     * Particularly bad in MPI communication (What\%):
    %       - Can be improved:
    %         * Fewer wait alls
    %         * Collected communication
    %     * Interface flux:
    %       - Setting/getting flux to/from the interfaces takes up a large chunk of time (What\%)
    %     * Actual transport is quite fast (What\%)
    %     * Number of segments improved for accurate result?
    %   - Compared to MIT still not great.
    \subsection{Conclusions: Spatial Decomposition}{\label{ssec:Conclusions:Spatial Decomposition}
      The use of graph partitioning for spatial decomposition in MPACT has been very successful.
      The methods have greatly improved user experience, as they allow for an arbitrary number of radial domains in 2D/1D calculations \footnote{or an arbitrary number of domains in 3-D \ac{MOC} calculations}.
      Unlike previous capabilities in MPACT, these methods also require no additional effort from the user.
      User feedback has been very positive on these methods, and the default partitioning in MPACT is now the \ac{REB} graph partitioning method.

      For 2-D (or 2D/1D) calculations, the three graph partitioning methods tested did not yield significantly different results.
      However, for general 3-D decompositions, the \ac{RSB} and \ac{RIB} methods generally performed better than the \ac{REB} method for low to medium numbers of domains.
      The \ac{REB} method showed significant advantage for highly decomposed cores.
      These methods greatly improved the parallel efficiency of the \ac{MOC} calculation (per iteration).

      However, as more spatial domains are used, the convergence rate decreases and more iterations are necessary.
      This problem is present in previous methods, but seems to be worsened by using the graph partitioning methods; this could be explained by more re-entrant rays becoming possible in these methods.
      Additionally, the parallel scaling of the \ac{CMFD} solvers in MPACT is worse than that of the \ac{MOC} solvers.
      In fact, for highly decomposed cores, the \ac{CMFD} acceleration takes up more run-time than the \ac{MOC}.
    }

    \subsection{Conclusions: Linear Source}{\label{ssec:Conclusions:Linear Source}
      The modified exponential function used to address the instability in \ac{LSMOC} calculations with near-void regions was successful.
      Problems that previously demonstrated instabilities were no longer unstable, even when using an interpolation table with the same accuracy as that used with \ac{FS} calculations.
      Additionally, the study performed on more accurate tables showed that using an interpolation table for a single function and computing other functions was more efficient than tabulating the three functions.
      This was able to improve the exponential calculation time by 9\% and the total \ac{MOC} run-time by 3\%, compared to the previous method of tabulating all three functions.

      The improved formulation of the \ac{LSA} was able to entirely eliminate the group-dependent coefficients by reducing them to only spatially dependent terms.
      In fact, these terms were reduced to a form of the same matrix used to transform from the spatial moments to the spatial expansion coefficients.
      It is the author's opinion that the form found in this work is more elegant than those previously presented, and does not introduce any additional operations (assuming a modern computer architecture, with \acf{FMA} instructions).
      The elimination of the group-dependence in these coefficients was able to reduce run-times in multiphysics calculations by up to 15\%, as the recomputation of these coefficients was no longer necessary.

      Mesh sensitivity studies were performed on several \ac{VERA} progression problems \cite{VERAProblems}.
      Coarse mesh parameters were found, that maintained accuracy, while reducing the number of source regions in realistic calculations by up to 85\%.
      On \ac{VERA} problem 9, a 3-D cycle depletion of a quarter core with \ac{TH} feedback showed 10\% improvement in total run-time, and 30\% reduction in memory.
      Previous studies have shown that the \ac{LSA} was able to improve run-times in neutronics only calculations \cite{Ferrer2016,Gunow2018};
      this work has shown that the use of the improved formulation allows for advantages over the \ac{FSA} even in cases with multiple physics.
    }

    \subsection{Conclusions: MacroRay}{\label{ssec:Conclusions:MacroRay}
      The macroray ray-tracing method is a novel ray-tracing method studied as part of this dissertation work.
      The aim of this method is to allow for maintained accuracy while reducing the number of track-segments generated during ray-tracing, and thus reducing the amount of work during computation.
      The underlying idea is to use advanced transverse integration methods to improve the integration accuracy within each region.
      A similar idea, the macroband, had been studied in 2-D, but no extension to 3-D has been made.

      The method was shown to reduce the number of track-segments necessary for calculations with thin absorber regions.
      Though, in benchmarks without these features, the method did not show significant advantages over traditional ray-tracing methods.
      Unlike traditional ray-tracing methods, this method exhibits near uniform convergence with number of segments.
      This means as the number of rays is increased, the user can expect that the result will become closer to the converged result.
      This is a notable example of the advantages of the macroray method.
      Additionally, because of the dependence on internal geometry for ray-tracing, rays are forced through thin regions;
        this means the method can be used as more of a ``black box'' solver, without requiring as much knowledge of the method by the user.
    }
  }

  \section{Future Work}{\label{sec:Future Work}
    \subsection{Future Work: Spatial Decomposition}{\label{ssec:Future Work:Spatial Decomposition}
      The spatial decomposition in MPACT has several avenues for future improvement.
      This most obvious is to improve the parallel efficiency of the \ac{CMFD} solvers; \ac{CMFD} begins to dominate run-time as large numbers of spatial domains are used.
      The use of a multigrid \ac{CMFD} solver \cite{Yee2018}, may reduce the total run-time fraction of \ac{CMFD} relative to what was found in this work, but the multigrid solver does not address the poor parallel scaling.
      The next is that, the methods of this work created and partitioned a graph that only considered the computational expenses of \ac{MOC}.
      As \ac{CMFD} becomes a significant component of run-time in highly decomposed cases, it is warranted to investigate constructing a graph that considers both \ac{MOC} and \ac{CMFD} costs.
      Finally, it may be possible to modify the graph partitioning methods to avoid ``jagged'' domains, which negatively impact convergence, without negatively affecting load-balance.
    }

    \subsection{Future Work: Linear Source}{\label{ssec:Future Work:Linear Source}
      The \ac{LSA} has been shown to allow for significantly coarser meshes while maintaining accuracy.
      While unlikely to improve run-times for 2-D calculations, some groups have begun to investigate quadratic axial sources for 3-D \ac{MOC} calculations. % [CITATION]
      The use of these methods would allow the axial mesh to be significantly coarsened.
      Due to the success of the \ac{LSA}, a quadratic axial source may be able to improve run-times for axially large calculations, such as reactors.
      It may also be possible to extend \citetp{Ferrer2012b} work for an arbitrary source-order \ac{MOC} for general geometries.

      Finally, the use of \acp{GPGPU} has been studied for \ac{FSMOC} calculations, but not for the \ac{LSMOC}. % [CITATION]
      Many recent computing clusters require codes to utilize \acp{GPGPU}; if this method is to see continued use, the efficient implementation on \acp{GPGPU} is a necessity.
      Even if clusters are ignored, \acp{GPGPU} allow for significant speed-up of \ac{MOC} calculations, and it is the author's opinion that all modern \ac{MOC} codes should utilize them.
    }

    \subsection{Future Work: MacroRay}{\label{ssec:Future Work:MacroRay}
      While the macroray was shown to reduce the number of segments for some cases, it is generally still slower than the solvers using the traditional ray-tracing methods.
      This is not entirely surprising, as the traditional solvers have been highly optimized over the past decade of the \ac{CASL} project, whereas the macroray solver was a first-pass implementation for this research.
      Nonetheless, if the method is to be used, the efficiency of the implementation must be improved; in particular, the MPI communication, and the cost of the interface approximations must be made more efficient.
      One possibility to improve these is to use a different approximation of the flux on the interfaces.
      While this work has been initial research into the use of the macroray method, if the method is to be considered for use in production, it will need to be implemented efficiently on \acp{GPGPU}.

      Additionally, one of the primary difficulties in this work was due to the choice of rectangular rays in the 2D-2D ray-tracing approach.
      The use of 2D-2D ray-tracing is common for \ac{MRT} method but may not be the best choice for macroray methods.
      Indeed, there is a good argument against its use due to the fact that rays are guaranteed not to ``span'' the problem's surfaces.
      A reasonable next step for this work would be to investigate the use of non-rectangular rays, which eliminate the need for some of the fix-ups used in this work.
    }
    % 1. Spatial Decomposition
    %   - Improved CMFD solvers (underway)
    %   - Deeper investigation into iterative inefficiencies (can we make this not as bad)
    % 2. Linear source
    %   - Quadratic axial source for 3-D MOC problems
    %   - Arbitrary higher-order source in MOC? (Generic extension of Ferrer's thesis work)
    %   - GPU
    % 3. MacroRay
    %   - Investigation of:
    %     * Non-rectangular rays
    %     * Other interface flux approximations
    %   - Improved efficiency:
    %     * MPI
    %     * Interface
    %     * Transport?
    %     * GPU?
  }

  % References
  \printbibliography
}