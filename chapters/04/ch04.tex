\chapter{Ray-Tracing}{\label{ch:Ray-Tracing}
    %%% Multi-group Energy quantities %%%
\DeclareDocumentCommand{\gprime}{}{g^{\prime}}
% Cross-sections
\DeclareDocumentCommand{\xs}{ O{t} O{\loc} O{g}}{\ensuremath{\CrossSection_{#1}^{#3}(#2)}}
\DeclareDocumentCommand{\xst}{ O{\loc} O{g} }{\xs[t][#1][#2]}
\DeclareDocumentCommand{\xsa}{ O{\loc} O{g} }{\xs[a][#1][#2]}
\DeclareDocumentCommand{\xsf}{ O{\loc} O{\gprime} }{\xs[f][#1][#2]}
\DeclareDocumentCommand{\xss}{ o O{\loc} O{\dirprime\vdot\dir} O{\gprime \to g} }{
    \IfNoValueOrEmptyTF{#1}
    {\xs[s][#2,#3][#4]}
    {\xs[s,#1][#2][#4]}
}
\DeclareDocumentCommand{\spect}{ O{\loc} O{g} }{\ensuremath{\Spectrum^{#2}(#1)}}
\DeclareDocumentCommand{\nufis}{ O{\loc} O{\gprime} }{ \ensuremath{\nu\xsf[#1][#2]}}
\DeclareDocumentCommand{\D}{ O{\loc} O{g} }{\ensuremath{D^{#2}(#1)}}

% Flux
\DeclareDocumentCommand{\aflux}{ O{\loc} O{\dir} O{g} }{\ensuremath{\AngularFlux^{#3}(#1,#2)}}
\DeclareDocumentCommand{\sflux}{ O{\loc} O{\gprime} }{\ensuremath{\ScalarFlux^{#2}(#1)}}
\DeclareDocumentCommand{\current}{ O{\loc} O{g} }{\ensuremath{\Current^{#2}(#1)}}
\DeclareDocumentCommand{\fluxmoma}{ O{\ell} O{n} O{\loc} O{g} }{\ensuremath{\ScalarFlux^{#2,#4}_{#1}(#3)}}

% Source
\DeclareDocumentCommand{\source}{ O{\loc} O{\dir} O{g} }{\ensuremath{q^{#3}(#1,#2)}}
\DeclareDocumentCommand{\sourcemoma}{ O{\ell} O{n} O{\loc} O{g} }{\ensuremath{q^{#2,#4}_{#1}(#3)}}

    %%% Discrete Ordinates Quantities %%%
\DeclareDocumentCommand{\mprime}{}{m^{\prime}}
\DeclareDocumentCommand{\dirm}{ O{m} }{\dir_{#1}}
\DeclareDocumentCommand{\wt}{ O{m} }{\Weight_{#1}}

% Quadrature set
\DeclareDocumentCommand{\angquad}{ O{N} }{ \mathcal{M}_{#1} }

% Cross-Sections
\DeclareDocumentCommand{\xss}{ o O{\loc} O{ m^{\prime}\!\to m} O{\gprime\!\to g} }{
    \IfNoValueOrEmptyTF{#1}
    {\xs[s,#3][#2][#4]}
    {\xs[s,#1][#2][#4]}
}

% Flux
\DeclareDocumentCommand{\aflux}{ O{\loc} O{m} O{g}}{\ensuremath{\AngularFlux^{#3}_{#2}\!\left(#1\right)}}

% Source
\DeclareDocumentCommand{\source}{ O{\loc} O{m} O{g} }{\ensuremath{q^{#3}_{#2}\!\left(#1\right)}}

    %%% MOC quantities %%%
% Geometric
\DeclareDocumentCommand{\Length}{}{s}
\DeclareDocumentCommand{\NormalizedLength}{}{t}
\DeclareDocumentCommand{\len}{ O{} }{\Length_{#1}}
\DeclareDocumentCommand{\segl}{ O{mki} }{\Length_{#1}}
\DeclareDocumentCommand{\nlen}{ O{m} }{\NormalizedLength_{#1}}
\DeclareDocumentCommand{\nsegl}{ O{mki} }{\NormalizedLength_{#1}}

\DeclareDocumentCommand{\centroid}{ O{\loc} O{i} }{#1_{#2}^{\text{c}}}
\DeclareDocumentCommand{\locIn}{ O{\loc} O{mki} }{{#1}_{#2}^{\text{in}}}
\DeclareDocumentCommand{\locOut}{ O{\loc} O{mki} }{{#1}_{#2}^{\text{out}}}
\DeclareDocumentCommand{\locCent}{ O{\loc} O{mki} }{{#1}_{#2}^{\text{c}}}
\DeclareDocumentCommand{\M}{ o O{i}}{%
    \IfNoValueOrEmptyTF{#1}
        {\vec{M}_{#2}}
        {M_{#2,#1}}
}
\DeclareDocumentCommand{\C}{o O{i} O{g}}{%
    \IfNoValueOrEmptyTF{#1}
        {\vec{C}_{#2}^{#3}}
        {C_{#2,#1}^{#3}}
}


% Integration
\DeclareAutoPairedDelimiter{\MOCTrackIntegral}{\langle}{\rangle_{mki}}
\DeclareAutoPairedDelimiter{\MOCSingleAngleIntegral}{\langle}{\rangle_{mi}}
\DeclareAutoPairedDelimiter{\MOCIntegral}{\langle}{\rangle_{i}}

% Cross-sections
\DeclareDocumentCommand{\xs}{ O{t} O{i} O{g}}{\ensuremath{\CrossSection_{#1,#2}^{#3}}}
\DeclareDocumentCommand{\xst}{ O{i} O{g} }{\xs[t][#1][#2]}
\DeclareDocumentCommand{\xsa}{ O{i} O{g} }{\xs[a][#1][#2]}
\DeclareDocumentCommand{\xsf}{ O{i} O{\gprime} }{\xs[f][#1][#2]}
\DeclareDocumentCommand{\xss}{ o O{i} O{m'\to m} O{\gprime \to g} }{
    \IfNoValueOrEmptyTF{#1}
    {\xs[s][#2,#3][#4]}
    {\xs[s,#1][#2][#4]}
}

\DeclareDocumentCommand{\spect}{ O{i} O{g} }{\ensuremath{\Spectrum^{#2}_{#1}}}
\DeclareDocumentCommand{\nufis}{ O{i} O{\gprime} }{ \ensuremath{\nu\xsf[#1][#2]}}
\DeclareDocumentCommand{\D}{ O{i} O{g} }{\ensuremath{D^{#2}_{#1}}}
\DeclareDocumentCommand{\opt}{ O{m} O{g} }{\OpticalThickness_{#1}^{#2}}
\DeclareDocumentCommand{\segopt}{ O{mki} O{g} }{\opt[#1][#2]}

% MOC Parameters
\DeclareDocumentCommand{\tA}{ O{a} }{\ensuremath{\delta\!A_{#1}}}
\DeclareDocumentCommand{\Weight}{}{w}
\DeclareDocumentCommand{\wt}{ O{m} }{\Weight_{#1}}
\DeclareDocumentCommand{\wtbar}{ O{m} }{\overline{\Weight}_{#1}}
\DeclareDocumentCommand{\renorm}{ O{i} }{\ensuremath{\xi_{#1}}}

% Flux
\DeclareDocumentCommand{\aflux}{ O{mki} O{g} O{\len} }{
    \IfNoValueOrEmptyTF{#3}
    {\AngularFlux^{#2}_{#1}}
    {\ensuremath{\AngularFlux^{#2}_{#1}\!\left(#3\right)}}
}
\DeclareDocumentCommand{\afluxin}{ O{mki} O{g} }{\AngularFlux^{#2,\text{in}}_{#1}}
\DeclareDocumentCommand{\afluxout}{ O{mki} O{g} }{\AngularFlux^{#2,\text{out}}_{#1}}
\DeclareDocumentCommand{\sflux}{ O{g} O{i} }{\ScalarFlux_{#2}^{#1}}
\DeclareDocumentCommand{\current}{ O{i} O{g} }{\Current^{#2}_{#1}}
\DeclareDocumentCommand{\tfluxF}{ O{mki} O{g} }{ \overline{\AngularFlux}_{#1}^{#2} }          % Average flux-moment along track
\DeclareDocumentCommand{\tfluxL}{ O{mki} O{g} }{  \widehat{\AngularFlux}_{#1}^{#2} }          % Linear flux-moment along track
\DeclareDocumentCommand{\dflux}{ O{mki} O{g} }{ \Delta\AngularFlux_{#1}^{#2} }                % Difference of angular flux along track
\DeclareDocumentCommand{\sfluxF}{ O{i} O{g} }{ \overline{\ScalarFlux}_{#1}^{#2} }             % Average scalar flux
% \DeclareDocumentCommand{\sfluxL}{ o O{i} O{g} }{ % Linear expansion coeff (Scalar Flux)
%     \IfNoValueOrEmptyTF{#1}
%         {\lvec{\widehat{\ScalarFlux}}_{#2}^{#3}}
%         {\widehat{\ScalarFlux}_{#2,#1}^{#3}}
% }
\DeclareDocumentCommand{\sfluxL}{ O{i} O{g} o }{ % Linear expansion coeff (Scalar Flux)
    \IfNoValueOrEmptyTF{#3}
        {\lvec{\widehat{\ScalarFlux}}_{#1}^{#2}}
        {\widehat{\ScalarFlux}_{#1,#3}^{#2}}
}
% \DeclareDocumentCommand{\sfluxL}{ m O{i} O{g} }{\widehat{\ScalarFlux}_{#2,#1}^{#3} }          % Linear expansion coeff (Scalar flux)
\DeclareDocumentCommand{\afluxmom}{ O{\ell} O{n} O{i} O{\gprime} }{\FluxMoment_{#3,#1}^{#4,#2}}

% Source
\DeclareDocumentCommand{\source}{ O{mki} O{g} O{\len} }{\ensuremath{q^{#2}_{#1}\!\left(#3\right)}}
\DeclareDocumentCommand{\tsrcF}{ O{mki} O{g} }{ \overline{q}_{#1}^{#2} }          % Average source along track
\DeclareDocumentCommand{\tsrcL}{ O{mi} O{g} }{ \widehat{q}_{#1}^{#2} }          % Linear source along track
\DeclareDocumentCommand{\src}{ O{i} O{g} }{ \Source_{#1}^{#2}}                          % Generic source
\DeclareDocumentCommand{\srcF}{ O{i} O{g} }{ q_{#1}^{#2} }             % Average source
\DeclareDocumentCommand{\srcL}{ o O{i} O{g} }{ % Linear expansion coeff (Source)
    \IfNoValueOrEmptyTF{#1}
        {\lvec{\widehat{q}}_{#2}^{#3}}
        {\widehat{q}_{#2,#1}^{#3}}
}

% Linear source operators / functions
\DeclareDocumentCommand{\FluxToSource}{ O{g} }{\mathcal{S}^{#1}}
    \def\figpath{chapters/04/figures/}
    \graphicspath{ {\figpath} }

    The \acf{MOC} \cite{Askew1972} is based on solving the transport equation along many characteristic tracks or rays.
    These rays are followed through the reactor geometry in a process generally referred to as ``ray-tracing''.
    The placement and storage of these tracks is significant with respect to both calculation accuracy as well as computational performance.
    This section servers to give an overview of the ray-tracing methods that have been developed, as well as motivate the use of the proposed 3-D ``MacroRay'' ray-tracing method.

    \section{Modular Ray-Tracing}{\label{sec:RT:Modular Ray-Tracing}
        The most straight-forward approach to perform a \ac{MOC} calculation is to create rays which span the domain of the transport problem being solved.
        However, in this approach, the information of each ray-segment must be stored, or computed on-the-fly.
        In large problems the number of ray-segments can become exceedingly large, and this approach is not feasible due to memory constraints.

        This led to the development of so-called ``modular'' ray-tracing methods \cite{Filippone1980,Saji2000,Wu2003,Kochunas2013}, in which the regularity of reactor designs is utilized to reduce memory usages.
        In typical reactor designs, certain geometries (like assemblies) are repeated throughout the core.
        Rather than laying tracks down for the global geometry, the transport problem is partitioned into ``modules'' which represent a small often repeated geometries in the problem.
        The ray-tracing data is generated for each module, in such a way there is direct linking of tracks on module interfaces; this is the \ac{DNPL} technique devised by \citet{Saji2000}.
        This significantly reduces the amount of ray-tracing data that needs to be stored in \ac{MOC} calculations, and has been widely adopted in \ac{MOC} transport codes \cite{Hong1998,Jung2009,Tang2009,DeCART,APOLLO3,MPACT2016,Hebert2017a}.
        Tracks spanning the global domain are then constructed by connecting multiple modular rays, as is depicted in \cref{fig:MRT:Modular Ray Tracing}.

        \begin{figure}[h]
            \centering
            \def\svgwidth{0.4\linewidth}
            \input{\figpath/ModularRayTracing.pdf_tex}
            \caption{Depiction of modular ray-tracing method. Global (long) rays can be constructed by connecting multiple modular rays, as is shown in red.}
            \label{fig:MRT:Modular Ray Tracing}
        \end{figure}

        The modular ray-tracing technique, with \ac{DNPL}, requires that the number of tracks on a modular boundary is an integer.
        This additionally requires that all modules have the same spatial dimensions, given by the pitches $P_x$ and $P_y$, and that the spacing between all tracks in a direction are constant.
        Let $\tA[a0]$ be the desired ray-spacing for an azimuthal angle $a$, and $\Azimuthal_{a0}$ be the desired azimuthal angle.
        The number of rays on the $x$ and $y$ module boundaries can be determined as
        \begin{subequations}\label{eqs:MRT:Nx/y}
            \begin{equation}\label{eq:MRT:Nx}
                N_x = \ceil*{\frac{P_x \sin(\Azimuthal_{a0})}{\tA[a0]}},
            \end{equation}
            and
            \begin{equation}\label{eq:MRT:Ny}
                N_y = \ceil*{\frac{P_x \cos(\Azimuthal_{a0})}{\tA[a0]}}.
            \end{equation}
        \end{subequations}
        The $x$ and $y$ distance between rays can be determined by
        \begin{subequations}\label{eqs:MRT:x/y ray-spacing}
            \begin{equation}\label{eq:MRT:x ray-spacing}
                \delta_x = \frac{P_x}{N_x},
            \end{equation}
            and
            \begin{equation}\label{eq:MRT:y ray-spacing}
                \delta_y = \frac{P_y}{N_y}.
            \end{equation}
        \end{subequations}
        The azimuthal angle is then ``corrected'' to represent the true angle at which the rays are placed,
        \begin{equation}\label{eq:MRT:Azimuthal Correction}
            \Azimuthal_{a} = \tan^{-1}\!\left(\frac{\delta_y}{\delta_x}\right),
        \end{equation}
        and the corrected ray-spacing is then given by
        \begin{equation}\label{eq:MRT:Corrected Radial Ray-Spacing}
            \tA[a] = \delta_x \sin(\Azimuthal_{a}).
        \end{equation}

        Due to the constraint of \ac{DNPL}, the directional quadrature is perturbed in the process of ray-tracing.
        As described in \cref{ssec:MOC:FSA:Particle Conservation}, this has implications on particle conservation in calculations.
        To maintain accuracy, a higher-order directional quadrature may need to be used would otherwise have been necessary.
    }
    \section{Mobile Chords}{\label{sec:RT:Mobile Chords}
        The mobile chord method was introduced by \citet{Villarino1992} in the HELIOS code for \ac{CP} calculations, and adapted to the \ac{MOC} by \citet{Yamamoto2008}.
        In the typical equidistant ray-tracing method, a ray is placed at the center of the ray-width.
        The mobile chord method offsets the ray from the center, with differing offsets in each direction.
        This has generally shown to be more accurate than the typical equidistant ray-tracing method \cite{Yamamoto2008}, but is not directly compatible with the \ac{DNPL} technique.
        While ray widths are still linked, the ray-traces are not; though, this does not seem to introduce significant discretization errors \cite{Yamamoto2008}.
    }
    \section{Macroband}{\label{sec:RT:Macroband}
        The \emph{macroband} method  was originally proposed by \citet{Villarino1992} for \ac{CP} calculations in HELIOS.
        In this method, characteristic rays placed within ``macrobands'' which are separated by tangential and intersection points in the mesh.
        There is no material or geometric discontinuities within each macroband segment (macrosegment), and thus the direction-of-flight averaged angular flux in each macrosegment is smooth with regards to the transverse direction.
        Since integration in the \ac{MOC} is akin to a quadrature integration, this indicates that a more advanced quadrature, for ray placement and width, can be used to reduce discretization error \cite{Yamamoto2005}.

       \begin{figure}[h]
            \centering
            \begin{subfigure}[t]{0.45\linewidth}
                \centering
                \def\svgwidth{0.85\linewidth}
                \input{\figpath/EquidistantRaySpacing.pdf_tex}
                \caption{Equidistant}
            \end{subfigure}%
            ~
            \begin{subfigure}[t]{0.45\linewidth}
                \centering
                \def\svgwidth{0.85\linewidth}
                \input{\figpath/MacroBandRays.pdf_tex}
                \caption{Macroband}
            \end{subfigure}
            \caption{Visualization of hypothetical tracks for (a) equidistant and (b) macroband ray-tracing methods. Boundaries between macrobands are shown as red dotted lines.}
            \label{fig:RT:Equidisant vs Macroband}
        \end{figure}

        Macrobands are determined by the computational mesh; for large heterogeneous assemblies, this results in very thin macrobands, which would result in significantly increased computation time.
        In the related \ac{CDP} \citet{Hong1999} proposed that macroband ray-tracing data only be generated on unique subsystems.
        Similarly, \citet{Yamamoto2005} proposed the \ac{MRMB} in which macroband ray-tracing data is only generated for unit-cells.
        These techniques are similar to the modular ray-tracing technique in that ray-tracing data is only generated for unique subsystems, which significantly reduces the amount of ray-tracing data.
        The macroband ray-tracing process is displayed for a single pin-cell mesh in \cref{fig:RT:Macroband Process}.

       \begin{figure}[!ht]
            \centering
            \begin{subfigure}[t]{0.33\linewidth}
                \centering
                \def\svgwidth{0.95\linewidth}
                \input{\figpath/MacroBandProcess1.pdf_tex}
                \caption{Determine intersection and tangent points}
            \end{subfigure}%
            \hfill
            \begin{subfigure}[t]{0.33\linewidth}
                \centering
                \def\svgwidth{0.95\linewidth}
                \input{\figpath/MacroBandProcess2.pdf_tex}
                \caption{Determine macroband boundaries (through identified points)}
            \end{subfigure}%
            \hfill
            \begin{subfigure}[t]{0.33\linewidth}
                \centering
                \def\svgwidth{0.95\linewidth}
                \input{\figpath/MacroBandProcess3.pdf_tex}
                \caption{Perform ray-tracing within macroband boundaries}
            \end{subfigure}
            \caption{Ray-tracing process for macroband.}
            \label{fig:RT:Macroband Process}
        \end{figure}

        However, these techniques are fundamentally incompatible with the \ac{DNPL} technique, at thus an approximation of angular flux must be made on subsystem interfaces.
        \citet{Yamamoto2005} proposed linearly interpolating the angular flux on cell boundaries, though other techniques involving averaging the angular flux on sub-boundaries have been utilized in the \ac{CDP} \cite{Liu2014}.
        Alternatively, these methods do not require adjustment to the angular quadrature.

        \citet{Yamamoto2005} found that the macroband method (using a Gauss-Legendre quadrature for ray placement), was more accurate than conventional ray-tracing methods with equidistant ray-spacing.
        \citet{Fevotte2007} proposed a new tracking technique similar to the macroband method, in which rays (placed equidistantly) are divided into sub-bands, which are effectively \emph{locally} projected macrobands, and the average flux of the sub-bands is propagated along each ray.
        Studies of ray-spacing with macroband, and \citetp{Fevotte2007} method, have indicated that coarser ray-spacing can be used while maintaining accuracy \cite{Yamamoto2005,Fevotte2007,Yamamoto2008}.
    }
    \section{Three-Dimensional Ray-Tracing Techniques}{\label{sec:RT:Three-Dimensional Ray-Tracing Techniques}
        The \ac{MOC} is naturally extended to three-dimensional calculations along characteristic tracks spanning three-dimensions.
        However, 3-D \ac{MOC} presents significant computational challenges.
        This has led to significant research effort into developing more efficient approaches to three-dimensional \ac{MOC} \cite{Kochunas2013,Tramm2016,Yamamoto2017,Graziano2017,Gunow2017}.
        One of the main focuses of this research has been in reducing the complexity introduced by three-dimensional ray-tracing.

        \subsection{3-D Modular Ray-Tracing}{\label{ssec:RT:3D Modular Ray-Tracing}
            In three-dimensional \ac{MOC} calculations, characteristic rays must be laid down through the three-dimensional domain;
            Generally, three-dimensional tracks are generated by first creating a set to two-dimensional tracks, and generating three-dimensional tracks that project onto these two-dimensional tracks \cite{Kochunas2013,Shaner2015}.
            The generation of these two-dimensional tracks can be simplified as viewing each of the two-dimensional tracks as a plane in the axial and characteristic directions, $z$ and $\len$ respectively.
            Three-dimensional tracks are then produced by performing two-dimensional ray-tracing on this characteristic plane \cite{Kochunas2013}, as is shown in \cref{fig:RT:3-D Ray-tracing}

            \begin{figure}[h]
                \centering
                \def\svgwidth{0.85\linewidth}
                \input{\figpath/3D-raytracing.pdf_tex}
                \caption{3-D ray-tracing process. Generate 2-D tracks, these become characteristic planes. Along each plane, perform 2-D ray-tracing. The highlighted (red) characteristic track in the 2-D pin-cell on the left becomes the characteristic plane on the right.}
                \label{fig:RT:3-D Ray-tracing}
            \end{figure}

            There are subtleties in the generation of three-dimensional tracks, which have led to the development of different 3-D modular ray-tracing techniques \cite{Kochunas2013,Shaner2015}.
            Previous work had reported that direct use of the \ac{MRT} method in 3-D, required that tracks be stored separately for the forward and backward directions \cite{Kochunas2013};
            this has, however, been shown not to be the case \cite{Shaner2015}.
            The simplified \ac{MRT} was developed to avoid this issue \cite{Kochunas2013}, but generates significantly more characteristic tracks \cite{Shaner2015}.

            As discussed in \cref{sec:RT:Modular Ray-Tracing}, 2-D \ac{MRT} perturbs the azimuthal angles in the directional quadrature to ensure \ac{DNPL}.
            For 3-D \ac{MRT}, this also perturbs the polar angles in the directional quadrature.
            \citet{Kochunas2013} also found that modularization of the directional quadratures led to clustering of the discrete directions, which introduces significant error in the integration of spherical harmonics moments \cite{Kochunas2013}, and has implications on particle conservation in anisotropic calculations (\cref{ssec:MOC:FSA:Particle Conservation}).
            In order to avoid this issue, the axial ray-spacing can be reduced until the modularized polar angle is only perturbed within some error criteria \cite{Kochunas2013}; however, this leads to significant increases in the number of tracks (and thus increases computational costs).
        }
        \subsection{Chord-Classification}{\label{ssec:RT:Chord-Classification}
            While \ac{MRT} significantly reduces the storage requirements of ray-tracing data, in realistic calculations, the data cannot fully be stored in a processors cache.
            Loading this data from main memory is slow, and in general, the movement of this ray-tracing data into different levels of cache uses significant amounts of time in these calculations.
            This led to the development of a technique called \emph{chord-classification} for locally axially extruded geometries \cite{Sciannandrone2016}.
            Chord-classification recognizes that reactors typically have regularities in the 3-D geometries; this allows for characteristic track-segments to be classified into sets which share the same length.
            As shown in \cref{fig:RT:Chord-Classification}, rays on the same characteristic plane that intersect two vertical mesh boundaries will have the same lengths.
            Similarly for rays which intersect horizontal planes (though, as axial meshes are usually tall, this is not as common).

            \begin{figure}[h]
                \centering
                \def\svgwidth{0.45\linewidth}
                \input{\figpath/ChordClassification.pdf_tex}
                \caption{3-D example of chord-classification. Colored (red and blue) characteristic tracks represent groups of ``V-chords'', rays between two vertical surfaces.}
                \label{fig:RT:Chord-Classification}
            \end{figure}

            By classifying chords of the same length, the length can be stored only once.
            Additionally, this means that the exponential functions ($F_1(\segopt)$ for \ac{FSMOC}), only need to be calculated once as well.
            This adds irregularity to the data accessing, which is generally not optimal for computation, and thus is only expected to improve calculation speeds if a large number of chords can be classified \cite{Sciannandrone2016}.
            Indeed, it was observed that most rays (96\%) could be categorized as ``V-chords'' which intersect two vertical surfaces, which led to a 40\% reduction in transport sweep time \cite{Sciannandrone2016}.
        }
        \subsection{On-the-Fly Ray-Tracing}{\label{ssec:RT:On-the-Fly Ray-Tracing}
            The on-the-fly ray-tracing technique \cite{Gunow2016} uses some of the ideas of the chord-classification method \cite{Sciannandrone2016}, and only stores two-dimensional track information.
            All three-dimensional tracks are generated and temporarily stored during the transport sweep (on-the-fly), leading to significant memory savings (94\% reduction), with minimal computational overhead \cite{Gunow2016}.
        }
        \subsection{Macroray}{\label{ssec:RT:Macroray}
            The \emph{macroray} method is a three-dimensional extension of the two-dimensional macroband method, and is a central contribution of this work.
            This method has been implemented as part of this work, and, to the best of the author's knowledge, has never been studied.
            The macroray is a extension of the macroband into three-dimensions, in which along each radial ray the macroband method is used to generate axial rays on the characteristic plane.
            This follows the general procedure used for three-dimensional ray-tracing, but a limitation of this approach is explained in more detail in \cref{sec:RT:Interface Flux Approximations}.
            In three-dimensions, rays can more generally be separated into ``parallel pipes'', to simplify notation these are referred to as macrorays.

            The motivation for investigating this method is for three primary reasons.
            The first, is that 2-D results \cite{Yamamoto2005,Fevotte2007,Yamamoto2008} have indicated that the macroband method allows for coarser ray-spacing (thus fewer rays) while maintaining accuracy.
            This is expected to increase computational efficiency, and lead to faster \ac{MOC} calculations.
            However, an extension to three-dimensions, if coarser ray-spacing can be used in both the axial and radial directions, the increase in computational efficiency is expected to be greater.

            Second, \ac{MRT} methods require adjustments to the directional quadrature, which is expected to decrease accuracy of numerical integrals over directions \cite{Kochunas2013}.
            The macroband and macroray methods require no such adjustment.
            Typically, it is the polar angle that has a more advanced quadrature (Gauss-Legendre, or Tabuchi-Yamamoto \cite{TabuchiYamamotoQuad}), and is thus more sensitive to perturbations than the azimuthal angles.
            Without the perturbation of the polar (and azimuthal) quadratures, it may be possible to maintain accuracy while using fewer directions than is possible with 3-D \ac{MRT} methods.

            Finally, \ac{MRT} methods require that the same ray-spacing parameters are used throughout the entire problem domain.
            Problems which have strong absorbers typically require a very fine mesh \cite{Fitzgerald2019}, which then requires that finer ray-spacing be used.
            This finer ray-spacing, which may only be required for a small percentage of the problem domain, is then used for the entire domain, leading to a significant increase in the number tracks.
            The macroband and macroray methods allow for use of different ray-spacing parameters in each subsystem; additionally, because macrobands are based on the computational mesh, an effectively finer ray-spacing will automatically be generated due to the fine spatial mesh.
            This allows for densely spaced tracks where they are necessary, but more coarsely spaced tracks where they are not; this is expected to lead to significant reduction in the number of tracks in such problems.

            Another consideration in three-dimensional locally axial extruded geometries is the chord-classification method.
            In the macroray method, rays are separated into the macrorays, which are guaranteed to be of the same class in the chord-classification method;
            this macroray classification is depicted in \cref{fig:RT:Chord-Classification Macroray}

            \begin{figure}[h]
                \centering
                \def\svgwidth{0.45\linewidth}
                \input{\figpath/ChordClassificationMacroBand.pdf_tex}
                \caption{3-D example of chord-classification with Macroray ray-tracing. Colored (red and blue) characteristic tracks represent groups of ``V-chords'', rays between two vertical surfaces.}
                \label{fig:RT:Chord-Classification Macroray}
            \end{figure}
        }
        \subsection{Other Approaches}{\label{ssec:RT:Other Approaches}
            \citet{Giho2008} proposed the axially simplified \ac{MOC} in three-dimensional calculations, which examines characteristic planes (two-dimensional tracks).
            Along these characteristic planes, an axially extruded geometry becomes a rectilinear mesh.
            The angular flux on the edges of these rectilinear cells is averaged, and a transport calculation is performed using the angular-dependent transmission probability method \cite{Yamamoto2015}.
            The \ac{LEAF} method \cite{Yamamoto2017} is an extension of the axially simplified \ac{MOC}, where the angular flux on cell boundaries is expanded in Legendre polynomials.
            In the \ac{LEAF} method, the source is also expanded in terms (up to second order) of Legendre polynomials.
        }
    }
    \section{Transport Sweeping with the Method of Characteristics}{\label{sec:RT:Transport Sweeping with the Method of Characteristics}
        The \ac{MOC} is used to iteratively solve the transport equation, the iterations are generally referred to as \emph{transport sweeps}.
        For given boundary conditions and source, a transport sweep is used to compute estimates of the scalar flux and other moments using equations in the form \cref{eqs:MOC:FSA:Region-Averaged Flux Moments} for \ac{FSMOC} and \cref{eqs:MOC:LSA:Region-Averaged Flux Moments} for \ac{LSMOC}.
        This is an overly generalized description of a transport sweep, and there are considerations that arise from the different ray-tracing techniques discussed.

        In global ray-tracking procedures, calculations are most often carried out by examining a ray from end to end.
        The angular flux at the ends can be found from the boundary conditions.
        Along each segment, a transmission calculation can be carried out (\cref{eqs:MOC:FSA:Angular Flux Solution,eqs:MOC:LSA:Angular Flux Solution}), and flux moments can be accumulated.
        This procedure is shared by the \ac{MRT} which constructs global rays (long rays) by linking modular rays.
        This allows for each ray calculation to be carried out in parallel, since each ray calculation is effectively independent \cite{Kochunas2013}; though special considerations must be taken to avoid race-conditions in the accumulation of moments.

        However, in the macroband and macroray methods, transport sweeping is carried out in a different manner.
        It is not possible to generate a continuous characteristic track spanning the global domain when using \ac{MRMB} techniques.
        Thus, transport calculations are carried out in a pin-by-pin (assuming pins are the subsystems on which track data is generated) basis.
        For each pin calculation, boundary or interface angular flux is loaded, and transmission/accumulation calculations are carried out by sweeping over the tracks.
        This can be done by considering each track separately, or by considering each macroray which are guaranteed to pass through the same regions.
        The angular flux on the exiting interface must be approximated, this can be done by interpolation \cite{Yamamoto2005}, or sub-boundary averaging \cite{Liu2014}.

        Pin-by-pin transport calculations must be carried out in a specific order, by considering the dependencies of angular flux.
        \Cref{fig:RT:Macroband Sweep Order} displays the sweeping order for a $2\times2$ array of pins in 2-D; this order is significantly different than the ray-by-ray order used in traditional \ac{MRT}-based \ac{MOC} calculations.
        Sweeping in this manner limits parallelism, though it may be possible to construct a dependency graph to get an optimal sweeping order considering macrorays individually, rather than on the pin-by-pin basis.

        \begin{figure}[h]
            \centering
            \def\svgwidth{0.65\linewidth}
            \input{\figpath/MacroBandSweepOrder.pdf_tex}
            \caption{Pin-by-pin sweeping order for a $2\times2$ domain with colors representing the order starting from the bottom left pin.}
            \label{fig:RT:Macroband Sweep Order}
        \end{figure}
    }
    \section{Interface Flux Approximations}{\label{sec:Interface Flux Approximations}
      The macroband and macroray ray-tracing methods are fundamentally incompatible with \acf{DNPL}.
      Although ray-tracing can be performed on smaller unique subsystems, similarly to the \ac{MRT} method, the rays on module interfaces are no longer guaranteed to be aligned.
      Approximations to the angular flux at these interfaces are thus necessary.
      \citet{Yamamoto2005} proposed performing linear interpolation of flux on these interfaces in the macroband method.
      However, linear interpolation, without adjustment, does not guarantee that the net current is conserved; which is an important aspect when considering \ac{CMFD} acceleration techniques.

      \citet{Liu2014} proposed a sub-boundary averaging method for use with 3-D \ac{CDP}.
      Although this technique was used in the context of a modular ray-tracing approach, it can be adapted to conserve net current on surfaces.
      The proposed sub-boundary averaging method partitioned surfaces into sub-boundaries.
      Each sub-boundary was assigned a single angular flux value, which was the average of the angular flux of all rays with centroids that intersected the sub-boundary.

      Only the centroid of the ray was considered, and not the actual shape of the intersection between the ray and surface.
      By not considering the shape of the ray, only a very fine ray-spacing can be used.
      If one considers the shape of the ray, and stores partial contributions to each surface, a coarser ray-spacing can be used, although this is a more expensive operation.
      This approach was the one taken for this work, and the details of this averaging scheme are examined in the subsequent subsections.

      %%%%%%%%%%%%%%%%%%%%%%%%%%%%%%%%%%%%%%%%%%%%%%%%%%%%%%%%%%%%%%%%%%%%%%%%%%%%
      % 2-D Sub-Boundary Averaging
      %%%%%%%%%%%%%%%%%%%%%%%%%%%%%%%%%%%%%%%%%%%%%%%%%%%%%%%%%%%%%%%%%%%%%%%%%%%%
      \subsection{2-D Sub-Boundary Averaging}{\label{ssec:RT:2-D Sub-Boundary Averaging}
        \blindtext[4]
      }
      %%%%%%%%%%%%%%%%%%%%%%%%%%%%%%%%%%%%%%%%%%%%%%%%%%%%%%%%%%%%%%%%%%%%%%%%%%%%
      % 3-D Sub-Boundary Averaging
      %%%%%%%%%%%%%%%%%%%%%%%%%%%%%%%%%%%%%%%%%%%%%%%%%%%%%%%%%%%%%%%%%%%%%%%%%%%%
      \subsection{3-D Sub-Boundary Averaging}{\label{ssec:3-D Sub-Boundary Averaging}
        \blindtext[8]
      }
    }
    % References
    % \printbibliography
}