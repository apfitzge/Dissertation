\chapter{MacroRay Three-Dimensional Ray-tracing Technique}{\label{ch:MacroRay Three-Dimensional Ray-tracing Technique}
  %%% Multi-group Energy quantities %%%
\DeclareDocumentCommand{\gprime}{}{g^{\prime}}
% Cross-sections
\DeclareDocumentCommand{\xs}{ O{t} O{\loc} O{g}}{\ensuremath{\CrossSection_{#1}^{#3}(#2)}}
\DeclareDocumentCommand{\xst}{ O{\loc} O{g} }{\xs[t][#1][#2]}
\DeclareDocumentCommand{\xsa}{ O{\loc} O{g} }{\xs[a][#1][#2]}
\DeclareDocumentCommand{\xsf}{ O{\loc} O{\gprime} }{\xs[f][#1][#2]}
\DeclareDocumentCommand{\xss}{ o O{\loc} O{\dirprime\vdot\dir} O{\gprime \to g} }{
    \IfNoValueOrEmptyTF{#1}
    {\xs[s][#2,#3][#4]}
    {\xs[s,#1][#2][#4]}
}
\DeclareDocumentCommand{\spect}{ O{\loc} O{g} }{\ensuremath{\Spectrum^{#2}(#1)}}
\DeclareDocumentCommand{\nufis}{ O{\loc} O{\gprime} }{ \ensuremath{\nu\xsf[#1][#2]}}
\DeclareDocumentCommand{\D}{ O{\loc} O{g} }{\ensuremath{D^{#2}(#1)}}

% Flux
\DeclareDocumentCommand{\aflux}{ O{\loc} O{\dir} O{g} }{\ensuremath{\AngularFlux^{#3}(#1,#2)}}
\DeclareDocumentCommand{\sflux}{ O{\loc} O{\gprime} }{\ensuremath{\ScalarFlux^{#2}(#1)}}
\DeclareDocumentCommand{\current}{ O{\loc} O{g} }{\ensuremath{\Current^{#2}(#1)}}
\DeclareDocumentCommand{\fluxmoma}{ O{\ell} O{n} O{\loc} O{g} }{\ensuremath{\ScalarFlux^{#2,#4}_{#1}(#3)}}

% Source
\DeclareDocumentCommand{\source}{ O{\loc} O{\dir} O{g} }{\ensuremath{q^{#3}(#1,#2)}}
\DeclareDocumentCommand{\sourcemoma}{ O{\ell} O{n} O{\loc} O{g} }{\ensuremath{q^{#2,#4}_{#1}(#3)}}

  %%% Discrete Ordinates Quantities %%%
\DeclareDocumentCommand{\mprime}{}{m^{\prime}}
\DeclareDocumentCommand{\dirm}{ O{m} }{\dir_{#1}}
\DeclareDocumentCommand{\wt}{ O{m} }{\Weight_{#1}}

% Quadrature set
\DeclareDocumentCommand{\angquad}{ O{N} }{ \mathcal{M}_{#1} }

% Cross-Sections
\DeclareDocumentCommand{\xss}{ o O{\loc} O{ m^{\prime}\!\to m} O{\gprime\!\to g} }{
    \IfNoValueOrEmptyTF{#1}
    {\xs[s,#3][#2][#4]}
    {\xs[s,#1][#2][#4]}
}

% Flux
\DeclareDocumentCommand{\aflux}{ O{\loc} O{m} O{g}}{\ensuremath{\AngularFlux^{#3}_{#2}\!\left(#1\right)}}

% Source
\DeclareDocumentCommand{\source}{ O{\loc} O{m} O{g} }{\ensuremath{q^{#3}_{#2}\!\left(#1\right)}}

  %%% MOC quantities %%%
% Geometric
\DeclareDocumentCommand{\Length}{}{s}
\DeclareDocumentCommand{\NormalizedLength}{}{t}
\DeclareDocumentCommand{\len}{ O{} }{\Length_{#1}}
\DeclareDocumentCommand{\segl}{ O{mki} }{\Length_{#1}}
\DeclareDocumentCommand{\nlen}{ O{m} }{\NormalizedLength_{#1}}
\DeclareDocumentCommand{\nsegl}{ O{mki} }{\NormalizedLength_{#1}}

\DeclareDocumentCommand{\centroid}{ O{\loc} O{i} }{#1_{#2}^{\text{c}}}
\DeclareDocumentCommand{\locIn}{ O{\loc} O{mki} }{{#1}_{#2}^{\text{in}}}
\DeclareDocumentCommand{\locOut}{ O{\loc} O{mki} }{{#1}_{#2}^{\text{out}}}
\DeclareDocumentCommand{\locCent}{ O{\loc} O{mki} }{{#1}_{#2}^{\text{c}}}
\DeclareDocumentCommand{\M}{ o O{i}}{%
    \IfNoValueOrEmptyTF{#1}
        {\vec{M}_{#2}}
        {M_{#2,#1}}
}
\DeclareDocumentCommand{\C}{o O{i} O{g}}{%
    \IfNoValueOrEmptyTF{#1}
        {\vec{C}_{#2}^{#3}}
        {C_{#2,#1}^{#3}}
}


% Integration
\DeclareAutoPairedDelimiter{\MOCTrackIntegral}{\langle}{\rangle_{mki}}
\DeclareAutoPairedDelimiter{\MOCSingleAngleIntegral}{\langle}{\rangle_{mi}}
\DeclareAutoPairedDelimiter{\MOCIntegral}{\langle}{\rangle_{i}}

% Cross-sections
\DeclareDocumentCommand{\xs}{ O{t} O{i} O{g}}{\ensuremath{\CrossSection_{#1,#2}^{#3}}}
\DeclareDocumentCommand{\xst}{ O{i} O{g} }{\xs[t][#1][#2]}
\DeclareDocumentCommand{\xsa}{ O{i} O{g} }{\xs[a][#1][#2]}
\DeclareDocumentCommand{\xsf}{ O{i} O{\gprime} }{\xs[f][#1][#2]}
\DeclareDocumentCommand{\xss}{ o O{i} O{m'\to m} O{\gprime \to g} }{
    \IfNoValueOrEmptyTF{#1}
    {\xs[s][#2,#3][#4]}
    {\xs[s,#1][#2][#4]}
}

\DeclareDocumentCommand{\spect}{ O{i} O{g} }{\ensuremath{\Spectrum^{#2}_{#1}}}
\DeclareDocumentCommand{\nufis}{ O{i} O{\gprime} }{ \ensuremath{\nu\xsf[#1][#2]}}
\DeclareDocumentCommand{\D}{ O{i} O{g} }{\ensuremath{D^{#2}_{#1}}}
\DeclareDocumentCommand{\opt}{ O{m} O{g} }{\OpticalThickness_{#1}^{#2}}
\DeclareDocumentCommand{\segopt}{ O{mki} O{g} }{\opt[#1][#2]}

% MOC Parameters
\DeclareDocumentCommand{\tA}{ O{a} }{\ensuremath{\delta\!A_{#1}}}
\DeclareDocumentCommand{\Weight}{}{w}
\DeclareDocumentCommand{\wt}{ O{m} }{\Weight_{#1}}
\DeclareDocumentCommand{\wtbar}{ O{m} }{\overline{\Weight}_{#1}}
\DeclareDocumentCommand{\renorm}{ O{i} }{\ensuremath{\xi_{#1}}}

% Flux
\DeclareDocumentCommand{\aflux}{ O{mki} O{g} O{\len} }{
    \IfNoValueOrEmptyTF{#3}
    {\AngularFlux^{#2}_{#1}}
    {\ensuremath{\AngularFlux^{#2}_{#1}\!\left(#3\right)}}
}
\DeclareDocumentCommand{\afluxin}{ O{mki} O{g} }{\AngularFlux^{#2,\text{in}}_{#1}}
\DeclareDocumentCommand{\afluxout}{ O{mki} O{g} }{\AngularFlux^{#2,\text{out}}_{#1}}
\DeclareDocumentCommand{\sflux}{ O{g} O{i} }{\ScalarFlux_{#2}^{#1}}
\DeclareDocumentCommand{\current}{ O{i} O{g} }{\Current^{#2}_{#1}}
\DeclareDocumentCommand{\tfluxF}{ O{mki} O{g} }{ \overline{\AngularFlux}_{#1}^{#2} }          % Average flux-moment along track
\DeclareDocumentCommand{\tfluxL}{ O{mki} O{g} }{  \widehat{\AngularFlux}_{#1}^{#2} }          % Linear flux-moment along track
\DeclareDocumentCommand{\dflux}{ O{mki} O{g} }{ \Delta\AngularFlux_{#1}^{#2} }                % Difference of angular flux along track
\DeclareDocumentCommand{\sfluxF}{ O{i} O{g} }{ \overline{\ScalarFlux}_{#1}^{#2} }             % Average scalar flux
% \DeclareDocumentCommand{\sfluxL}{ o O{i} O{g} }{ % Linear expansion coeff (Scalar Flux)
%     \IfNoValueOrEmptyTF{#1}
%         {\lvec{\widehat{\ScalarFlux}}_{#2}^{#3}}
%         {\widehat{\ScalarFlux}_{#2,#1}^{#3}}
% }
\DeclareDocumentCommand{\sfluxL}{ O{i} O{g} o }{ % Linear expansion coeff (Scalar Flux)
    \IfNoValueOrEmptyTF{#3}
        {\lvec{\widehat{\ScalarFlux}}_{#1}^{#2}}
        {\widehat{\ScalarFlux}_{#1,#3}^{#2}}
}
% \DeclareDocumentCommand{\sfluxL}{ m O{i} O{g} }{\widehat{\ScalarFlux}_{#2,#1}^{#3} }          % Linear expansion coeff (Scalar flux)
\DeclareDocumentCommand{\afluxmom}{ O{\ell} O{n} O{i} O{\gprime} }{\FluxMoment_{#3,#1}^{#4,#2}}

% Source
\DeclareDocumentCommand{\source}{ O{mki} O{g} O{\len} }{\ensuremath{q^{#2}_{#1}\!\left(#3\right)}}
\DeclareDocumentCommand{\tsrcF}{ O{mki} O{g} }{ \overline{q}_{#1}^{#2} }          % Average source along track
\DeclareDocumentCommand{\tsrcL}{ O{mi} O{g} }{ \widehat{q}_{#1}^{#2} }          % Linear source along track
\DeclareDocumentCommand{\src}{ O{i} O{g} }{ \Source_{#1}^{#2}}                          % Generic source
\DeclareDocumentCommand{\srcF}{ O{i} O{g} }{ q_{#1}^{#2} }             % Average source
\DeclareDocumentCommand{\srcL}{ o O{i} O{g} }{ % Linear expansion coeff (Source)
    \IfNoValueOrEmptyTF{#1}
        {\lvec{\widehat{q}}_{#2}^{#3}}
        {\widehat{q}_{#2,#1}^{#3}}
}

% Linear source operators / functions
\DeclareDocumentCommand{\FluxToSource}{ O{g} }{\mathcal{S}^{#1}}
  \def\figpath{chapters/MacroRay/figures/}
  \graphicspath{ {\figpath} }

  \section{Macroray}{\label{sec:RT:Macroray}
    The \emph{macroray} method is a three-dimensional extension of the two-dimensional macroband method, and is a central contribution of this work.
    This method has been implemented as part of this work, and, to the best of the author's knowledge, has never been studied.
    The macroray is a extension of the macroband into three-dimensions, in which along each radial ray the macroband method is used to generate axial rays on the characteristic plane.
    This follows the general procedure used for three-dimensional ray-tracing, but a limitation of this approach is explained in more detail in \cref{sec:RT:Interface Flux Approximations}.
    In three-dimensions, rays can more generally be separated into ``parallel pipes'', to simplify notation these are referred to as macrorays.

    The motivation for investigating this method is for three primary reasons.
    The first, is that 2-D results \cite{Yamamoto2005,Fevotte2007,Yamamoto2008} have indicated that the macroband method allows for coarser ray-spacing (thus fewer rays) while maintaining accuracy.
    This is expected to increase computational efficiency, and lead to faster \ac{MOC} calculations.
    However, an extension to three-dimensions, if coarser ray-spacing can be used in both the axial and radial directions, the increase in computational efficiency is expected to be greater.

    Second, \ac{MRT} methods require adjustments to the directional quadrature, which is expected to decrease accuracy of numerical integrals over directions \cite{Kochunas2013}.
    The macroband and macroray methods require no such adjustment.
    Typically, it is the polar angle that has a more advanced quadrature (Gauss-Legendre, or Tabuchi-Yamamoto \cite{TabuchiYamamotoQuad}), and is thus more sensitive to perturbations than the azimuthal angles.
    Without the perturbation of the polar (and azimuthal) quadratures, it may be possible to maintain accuracy while using fewer directions than is possible with 3-D \ac{MRT} methods.

    Finally, \ac{MRT} methods require that the same ray-spacing parameters are used throughout the entire problem domain.
    Problems which have strong absorbers typically require a very fine mesh \cite{Fitzgerald2019}, which then requires that finer ray-spacing be used.
    This finer ray-spacing, which may only be required for a small percentage of the problem domain, is then used for the entire domain, leading to a significant increase in the number tracks.
    The macroband and macroray methods allow for use of different ray-spacing parameters in each subsystem; additionally, because macrobands are based on the computational mesh, an effectively finer ray-spacing will automatically be generated due to the fine spatial mesh.
    This allows for densely spaced tracks where they are necessary, but more coarsely spaced tracks where they are not; this is expected to lead to significant reduction in the number of tracks in such problems.

    Another consideration in three-dimensional locally axial extruded geometries is the chord-classification method.
    In the macroray method, rays are separated into the macrorays, which are guaranteed to be of the same class in the chord-classification method;
    this macroray classification is depicted in \cref{fig:RT:Chord-Classification Macroray}

    \begin{figure}[h]
      \centering
      \def\svgwidth{0.45\linewidth}
      \input{\figpath/ChordClassificationMacroBand.pdf_tex}
      \caption{3-D example of chord-classification with Macroray ray-tracing. Colored (red and blue) characteristic tracks represent groups of ``V-chords'', rays between two vertical surfaces.}
      \label{fig:RT:Chord-Classification Macroray}
    \end{figure}
  }

  % References
  \printbibliography
}