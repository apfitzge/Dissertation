%%%%%%%%%%%%%%%%%%%%%%%%%%%%%%%%%%%%
% Aligned equation
\NewEnviron{aequation}{%
    \begin{equation}\begin{split}
        \BODY
    \end{split}\end{equation}
}

% Xparse extension
\newcommand{\DeclareAutoPairedDelimiter}[3]{%
  \expandafter\DeclarePairedDelimiter\csname Auto\string#1\endcsname{#2}{#3}%
  \DeclareRobustCommand{#1}{\csname Auto\string#1\endcsname*}}
\ExplSyntaxOn
\DeclareExpandableDocumentCommand{\IfNoValueOrEmptyTF}{mmm}{
  \IfNoValueTF{#1}{#2}{\tl_if_empty:nTF {#1} {#2} {#3}}
}
\ExplSyntaxOff
%%% General Mathematics %%%
\newcommand{\defined}{\equiv}
\newcommand{\pluseq}{\mathrel{+}=}        % Accumulation (plus-equals)
\DeclareDocumentCommand{\avg}{ m }{\overline{#1}}
\DeclareDocumentCommand{\vec}{ m }{\boldsymbol{#1}}             % Bold vectors
\DeclareDocumentCommand{\lvec}{ m }{\underline{\vec{#1}}}       % underline bold vectors
\DeclareDocumentCommand{\uvec}{ m }{\vec{\widehat{#1}}}         % Unit-vector
\DeclareDocumentCommand{\mat}{ m }{\boldsymbol{#1}}
\newcommand{\vdot}{\boldsymbol{\cdot}}           % Bold dot-mulitply
\newcommand{\grad}{\vec{\nabla}}
\newcommand{\dif}{\mathop{}\!\mathrm{d}}  % d for integrals
\DeclareDocumentCommand{\deriv}{ m m }{                   % Derivatives
  \frac{\dif{#1}}{\dif{#2}}
}
\DeclareDocumentCommand{\pderiv}{ m m }{                  % Partial derivatives
  \frac{\partial{#1}}{\partial{#2}}
}
\DeclarePairedDelimiter{\abs}{\lvert}{\rvert}       % Absolute value
\DeclarePairedDelimiter{\ceil}{\lceil}{\rceil}      % Ceiling
\DeclarePairedDelimiter{\floor}{\lfloor}{\rfloor}   % Floor
\DeclareAutoPairedDelimiter{\lrabs}{\lvert}{\rvert}       % Absolute value
\DeclareAutoPairedDelimiter{\lrceil}{\lceil}{\rceil}      % Ceiling
\DeclareAutoPairedDelimiter{\lrfloor}{\lfloor}{\rfloor}   % Floor
%%% Mathematical Operators %%%
\DeclareDocumentCommand{\intl}{ o o }{    % Integral w/ or w/o limits
  \IfValueTF{#1}
  {
    \IfValueTF{#2}
    {\int_{#1}^{#2}}
    {\int_{#1}}
  }
  {\int}
}
\DeclareDocumentCommand{\suml}{ o o }{    % Sum w/ or w/o limits
  \IfValueTF{#1}
  {
    \IfValueTF{#2}
    {\mathchoice{\sum\limits_{#1}^{#2}}{\sum\limits_{#1}^{#2}}{\sum_{#1}^{#2}}{\sum_{#1}^{#2}}}
    {\mathchoice{\sum\limits_{#1}}{\sum\limits_{#1}}{\sum_{#1}}{\sum_{#1}}}
  }
  {\sum}
}
\DeclareDocumentCommand{\prodl}{ o o }{    % Product w/ or w/o limits
  \IfValueTF{#1}
  {
    \IfValueTF{#2}
    {\mathchoice{\prod\limits_{#1}^{#2}}{\prod\limits_{#1}^{#2}}{\prod_{#1}^{#2}}{\prod_{#1}^{#2}}}
    {\mathchoice{\prod\limits_{#1}}{\prod\limits_{#1}}{\prod_{#1}}{\prod_{#1}}}
  }
  {\prod}
}
